\documentclass{article}
\usepackage{graphicx}
\usepackage{amsmath}

\title{Predicting MBA Admissions Using the SEMMA Methodology}
\author{}
\date{}

\begin{document}

\maketitle

\begin{abstract}
This research paper explores the application of the SEMMA (Sample, Explore, Modify, Model, Assess) methodology to predict MBA admissions outcomes using a dataset of student applications. We employed various machine learning techniques to develop and optimize predictive models, with the goal of identifying key factors that influence admission decisions. The steps of SEMMA are discussed in detail, from sampling and exploratory data analysis to model building and optimization. The Logistic Regression model emerged as the best performer, yielding an accuracy of 84.42\%, with strong recall for predicting denied applications. Challenges related to class imbalance, particularly in predicting waitlisted applicants, were also identified.
\end{abstract}

\section{Introduction}

MBA programs are highly selective, and institutions receive thousands of applications annually. Predicting which students are likely to be admitted, denied, or waitlisted can assist admission committees in making data-driven decisions. This paper follows the SEMMA methodology—a systematic data science framework consisting of Sample, Explore, Modify, Model, and Assess steps—to analyze a dataset of MBA applicants and build a predictive model for admissions decisions.

The dataset includes student information such as GPA, GMAT scores, work experience, major, race, and final admission outcomes. Our objective is to develop a model that can accurately predict whether a student is admitted, denied, or waitlisted.

\section{Methodology}

\subsection{Sampling}
The dataset comprises over 6000 records of MBA applicants. To make the computational process more efficient, a sample of 1000 entries was randomly selected. The sampling process was designed to maintain a representative distribution of the admission outcomes while ensuring computational efficiency.

\begin{table}[h]
\centering
\begin{tabular}{|c|c|}
\hline
Admission Outcome & Number of Samples \\
\hline
Admitted          & 156               \\
Denied            & 830               \\
Waitlisted        & 14                \\
\hline
\end{tabular}
\caption{Admission Outcomes in Sample}
\end{table}

\subsection{Exploration}

Exploratory Data Analysis (EDA) was conducted to better understand the structure and characteristics of the dataset. This step included generating summary statistics, identifying missing values, and plotting the distribution of various features.

\begin{figure}[h]
    \centering
    \includegraphics[width=0.8\textwidth]{admission_status_distribution.png}
    \caption{GMAT Score Distribution}
\end{figure}


\subsubsection{Numerical Features}
\begin{itemize}
    \item GPA: Mean GPA in the dataset is around 3.25. The distribution is slightly skewed towards higher GPAs.
    \item GMAT: GMAT scores range from 570 to 780, with a mean score of 650.
    \item Work Experience: Applicants generally have between 2 and 8 years of work experience, with a mean of approximately 5 years.
\end{itemize}

\begin{figure}[h]
    \centering
    \includegraphics[width=0.8\textwidth]{correlation_mba.png}
    \caption{GMAT Score Distribution}
\end{figure}


\subsubsection{Distribution Plots}
Below is a histogram showing the distribution of GMAT scores:

\begin{figure}[h]
    \centering
    \includegraphics[width=0.8\textwidth]{gmat_distribution.png}
    \caption{GMAT Score Distribution}
\end{figure}

\subsubsection{Categorical Features}
\begin{itemize}
    \item Major: The dataset is dominated by applicants from Business and STEM fields, with a smaller representation from Humanities.
    \item Race: There are some missing values in the race column, indicating a need for imputation or adjustment.
    \item Admission: The dataset has a significant class imbalance, with Denied applications making up the majority.
\end{itemize}

\subsubsection{Missing Values}
\begin{itemize}
    \item Race: Approximately 300 missing values.
    \item Admission: 830 missing values, which correspond to denied applications.
\end{itemize}

\subsection{Modification}

In this step, we cleaned the data, addressed missing values, performed feature encoding, and scaled the numerical features.

\subsubsection{Handling Missing Values}
\begin{itemize}
    \item Race: Missing values were replaced with the category 'Unknown'.
    \item Admission: Missing values in the admission column were replaced with 'Denied', as agreed during data exploration.
\end{itemize}

\subsubsection{Encoding Categorical Variables}
Categorical features like gender, international, major, race, work\_industry, and admission were label-encoded into numerical representations to be used in modeling. For example:

\begin{verbatim}
le = LabelEncoder()
for col in ['gender', 'international', 'major', 'race', 'work_industry', 'admission']:
    mba_sample[col] = le.fit_transform(mba_sample[col])
\end{verbatim}

\subsubsection{Feature Scaling}
Numerical features such as GPA, GMAT, and work experience were standardized to ensure they contribute equally to the model:

\begin{verbatim}
scaler = StandardScaler()
mba_sample[['gpa', 'gmat', 'work_exp']] = scaler.fit_transform(mba_sample[['gpa', 'gmat', 'work_exp']])
\end{verbatim}

\begin{figure}[h]
    \centering
    \includegraphics[width=0.8\textwidth]{feature_importance_mba.png}
    \caption{GMAT Score Distribution}
\end{figure}


\subsubsection{Outlier Detection}
Using the Interquartile Range (IQR) method, we detected and removed 7 outliers in the GPA column.

\subsection{Modeling}

In this step, we built multiple classification models to predict MBA admissions outcomes. The models tested included:
\begin{itemize}
    \item Logistic Regression
    \item Decision Trees
    \item Random Forest
    \item Support Vector Machines
    \item Gradient Boosting
\end{itemize}

\subsubsection{Baseline Model: Logistic Regression}
The baseline Logistic Regression model achieved an accuracy of 83.42\%, with strong precision and recall for predicting Denied applications.

\begin{table}[h]
\centering
\begin{tabular}{|c|c|c|c|}
\hline
Metric    & Admitted & Denied  & Waitlisted \\
\hline
Precision & 53.33\%  & 85.87\% & 0\%        \\
Recall    & 25.81\%  & 95.76\% & 0\%        \\
F1 Score  & 34.78\%  & 90.54\% & 0\%        \\
Accuracy  &          &         & 83.42\%    \\
\hline
\end{tabular}
\caption{Logistic Regression Performance Metrics}
\end{table}

\subsubsection{Comparison of Models}
Several models were compared based on their performance metrics:

\begin{table}[h]
\centering
\begin{tabular}{|c|c|c|c|c|}
\hline
Model                 & Accuracy & Precision (Denied) & Recall (Denied) & F1 Score (Denied) \\
\hline
Logistic Regression    & 83.42\%  & 85.87\%            & 95.76\%         & 90.54\%            \\
Decision Tree          & 76.38\%  & 84.09\%            & 89.70\%         & 86.80\%            \\
Random Forest          & 81.91\%  & 86.52\%            & 93.33\%         & 89.80\%            \\
Support Vector Machine & 82.91\%  & 82.91\%            & 100.00\%        & 90.66\%            \\
Gradient Boosting      & 79.90\%  & 85.31\%            & 91.52\%         & 88.30\%            \\
\hline
\end{tabular}
\caption{Model Comparison}
\end{table}

Logistic Regression consistently outperformed other models, particularly for predicting denied applications.

\subsection{Assessment}

To further optimize the Logistic Regression model, we conducted hyperparameter tuning using GridSearchCV. We explored the impact of the regularization strength parameter (C) and solver selection.

\subsubsection{Hyperparameter Tuning Results}
The optimal parameters for the Logistic Regression model were:
\begin{itemize}
    \item C = 0.1 (Regularization strength)
    \item Solver = 'lbfgs'
\end{itemize}

The optimized model achieved an accuracy of 84.42\% with strong predictive power for denied applications.

\subsubsection{Cross-Validation}
Using 5-fold cross-validation, the mean accuracy was 83.89\%, demonstrating the model's reliability.

\section{Results}

The optimized Logistic Regression model outperformed other models, providing reliable predictions for MBA admissions, particularly for denied applications. However, challenges remained in predicting waitlisted applicants due to class imbalance.

\begin{table}[h]
\centering
\begin{tabular}{|c|c|}
\hline
Metric     & Score    \\
\hline
Accuracy   & 84.42\%  \\
Precision  & 86.02\%  \\
Recall     & 96.97\%  \\
F1 Score   & 91.17\%  \\
\hline
\end{tabular}
\caption{Optimized Model Performance}
\end{table}

The model showed consistent performance across all five folds in cross-validation, confirming its stability.

\section{Conclusion}

The application of the SEMMA methodology to predict MBA admissions yielded a strong model with high predictive accuracy for denied applicants. The Logistic Regression model, after optimization, emerged as the best performer. Class imbalance remained a key challenge, particularly for predicting waitlisted applicants, which suggests that future work should focus on techniques like SMOTE or class-weighted loss functions.

\subsection{Future Work}
\begin{itemize}
    \item \textbf{Handling Class Imbalance:} Explore methods such as oversampling or undersampling to improve prediction of minority classes.
    \item \textbf{Feature Engineering:} Additional domain knowledge could enhance feature selection and model performance.
\end{itemize}

This study demonstrates the efficacy of the SEMMA approach in developing practical, data-driven solutions for MBA admissions.

\end{document}

